
% Default to the notebook output style

    


% Inherit from the specified cell style.




    
\documentclass[11pt]{article}

    
    
    \usepackage[T1]{fontenc}
    % Nicer default font (+ math font) than Computer Modern for most use cases
    \usepackage{mathpazo}

    % Basic figure setup, for now with no caption control since it's done
    % automatically by Pandoc (which extracts ![](path) syntax from Markdown).
    \usepackage{graphicx}
    % We will generate all images so they have a width \maxwidth. This means
    % that they will get their normal width if they fit onto the page, but
    % are scaled down if they would overflow the margins.
    \makeatletter
    \def\maxwidth{\ifdim\Gin@nat@width>\linewidth\linewidth
    \else\Gin@nat@width\fi}
    \makeatother
    \let\Oldincludegraphics\includegraphics
    % Set max figure width to be 80% of text width, for now hardcoded.
    \renewcommand{\includegraphics}[1]{\Oldincludegraphics[width=.8\maxwidth]{#1}}
    % Ensure that by default, figures have no caption (until we provide a
    % proper Figure object with a Caption API and a way to capture that
    % in the conversion process - todo).
    \usepackage{caption}
    \DeclareCaptionLabelFormat{nolabel}{}
    \captionsetup{labelformat=nolabel}

    \usepackage{adjustbox} % Used to constrain images to a maximum size 
    \usepackage{xcolor} % Allow colors to be defined
    \usepackage{enumerate} % Needed for markdown enumerations to work
    \usepackage{geometry} % Used to adjust the document margins
    \usepackage{amsmath} % Equations
    \usepackage{amssymb} % Equations
    \usepackage{textcomp} % defines textquotesingle
    % Hack from http://tex.stackexchange.com/a/47451/13684:
    \AtBeginDocument{%
        \def\PYZsq{\textquotesingle}% Upright quotes in Pygmentized code
    }
    \usepackage{upquote} % Upright quotes for verbatim code
    \usepackage{eurosym} % defines \euro
    \usepackage[mathletters]{ucs} % Extended unicode (utf-8) support
    \usepackage[utf8x]{inputenc} % Allow utf-8 characters in the tex document
    \usepackage{fancyvrb} % verbatim replacement that allows latex
    \usepackage{grffile} % extends the file name processing of package graphics 
                         % to support a larger range 
    % The hyperref package gives us a pdf with properly built
    % internal navigation ('pdf bookmarks' for the table of contents,
    % internal cross-reference links, web links for URLs, etc.)
    \usepackage{hyperref}
    \usepackage{longtable} % longtable support required by pandoc >1.10
    \usepackage{booktabs}  % table support for pandoc > 1.12.2
    \usepackage[inline]{enumitem} % IRkernel/repr support (it uses the enumerate* environment)
    \usepackage[normalem]{ulem} % ulem is needed to support strikethroughs (\sout)
                                % normalem makes italics be italics, not underlines
    

    
    
    % Colors for the hyperref package
    \definecolor{urlcolor}{rgb}{0,.145,.698}
    \definecolor{linkcolor}{rgb}{.71,0.21,0.01}
    \definecolor{citecolor}{rgb}{.12,.54,.11}

    % ANSI colors
    \definecolor{ansi-black}{HTML}{3E424D}
    \definecolor{ansi-black-intense}{HTML}{282C36}
    \definecolor{ansi-red}{HTML}{E75C58}
    \definecolor{ansi-red-intense}{HTML}{B22B31}
    \definecolor{ansi-green}{HTML}{00A250}
    \definecolor{ansi-green-intense}{HTML}{007427}
    \definecolor{ansi-yellow}{HTML}{DDB62B}
    \definecolor{ansi-yellow-intense}{HTML}{B27D12}
    \definecolor{ansi-blue}{HTML}{208FFB}
    \definecolor{ansi-blue-intense}{HTML}{0065CA}
    \definecolor{ansi-magenta}{HTML}{D160C4}
    \definecolor{ansi-magenta-intense}{HTML}{A03196}
    \definecolor{ansi-cyan}{HTML}{60C6C8}
    \definecolor{ansi-cyan-intense}{HTML}{258F8F}
    \definecolor{ansi-white}{HTML}{C5C1B4}
    \definecolor{ansi-white-intense}{HTML}{A1A6B2}

    % commands and environments needed by pandoc snippets
    % extracted from the output of `pandoc -s`
    \providecommand{\tightlist}{%
      \setlength{\itemsep}{0pt}\setlength{\parskip}{0pt}}
    \DefineVerbatimEnvironment{Highlighting}{Verbatim}{commandchars=\\\{\}}
    % Add ',fontsize=\small' for more characters per line
    \newenvironment{Shaded}{}{}
    \newcommand{\KeywordTok}[1]{\textcolor[rgb]{0.00,0.44,0.13}{\textbf{{#1}}}}
    \newcommand{\DataTypeTok}[1]{\textcolor[rgb]{0.56,0.13,0.00}{{#1}}}
    \newcommand{\DecValTok}[1]{\textcolor[rgb]{0.25,0.63,0.44}{{#1}}}
    \newcommand{\BaseNTok}[1]{\textcolor[rgb]{0.25,0.63,0.44}{{#1}}}
    \newcommand{\FloatTok}[1]{\textcolor[rgb]{0.25,0.63,0.44}{{#1}}}
    \newcommand{\CharTok}[1]{\textcolor[rgb]{0.25,0.44,0.63}{{#1}}}
    \newcommand{\StringTok}[1]{\textcolor[rgb]{0.25,0.44,0.63}{{#1}}}
    \newcommand{\CommentTok}[1]{\textcolor[rgb]{0.38,0.63,0.69}{\textit{{#1}}}}
    \newcommand{\OtherTok}[1]{\textcolor[rgb]{0.00,0.44,0.13}{{#1}}}
    \newcommand{\AlertTok}[1]{\textcolor[rgb]{1.00,0.00,0.00}{\textbf{{#1}}}}
    \newcommand{\FunctionTok}[1]{\textcolor[rgb]{0.02,0.16,0.49}{{#1}}}
    \newcommand{\RegionMarkerTok}[1]{{#1}}
    \newcommand{\ErrorTok}[1]{\textcolor[rgb]{1.00,0.00,0.00}{\textbf{{#1}}}}
    \newcommand{\NormalTok}[1]{{#1}}
    
    % Additional commands for more recent versions of Pandoc
    \newcommand{\ConstantTok}[1]{\textcolor[rgb]{0.53,0.00,0.00}{{#1}}}
    \newcommand{\SpecialCharTok}[1]{\textcolor[rgb]{0.25,0.44,0.63}{{#1}}}
    \newcommand{\VerbatimStringTok}[1]{\textcolor[rgb]{0.25,0.44,0.63}{{#1}}}
    \newcommand{\SpecialStringTok}[1]{\textcolor[rgb]{0.73,0.40,0.53}{{#1}}}
    \newcommand{\ImportTok}[1]{{#1}}
    \newcommand{\DocumentationTok}[1]{\textcolor[rgb]{0.73,0.13,0.13}{\textit{{#1}}}}
    \newcommand{\AnnotationTok}[1]{\textcolor[rgb]{0.38,0.63,0.69}{\textbf{\textit{{#1}}}}}
    \newcommand{\CommentVarTok}[1]{\textcolor[rgb]{0.38,0.63,0.69}{\textbf{\textit{{#1}}}}}
    \newcommand{\VariableTok}[1]{\textcolor[rgb]{0.10,0.09,0.49}{{#1}}}
    \newcommand{\ControlFlowTok}[1]{\textcolor[rgb]{0.00,0.44,0.13}{\textbf{{#1}}}}
    \newcommand{\OperatorTok}[1]{\textcolor[rgb]{0.40,0.40,0.40}{{#1}}}
    \newcommand{\BuiltInTok}[1]{{#1}}
    \newcommand{\ExtensionTok}[1]{{#1}}
    \newcommand{\PreprocessorTok}[1]{\textcolor[rgb]{0.74,0.48,0.00}{{#1}}}
    \newcommand{\AttributeTok}[1]{\textcolor[rgb]{0.49,0.56,0.16}{{#1}}}
    \newcommand{\InformationTok}[1]{\textcolor[rgb]{0.38,0.63,0.69}{\textbf{\textit{{#1}}}}}
    \newcommand{\WarningTok}[1]{\textcolor[rgb]{0.38,0.63,0.69}{\textbf{\textit{{#1}}}}}
    
    
    % Define a nice break command that doesn't care if a line doesn't already
    % exist.
    \def\br{\hspace*{\fill} \\* }
    % Math Jax compatability definitions
    \def\gt{>}
    \def\lt{<}
    % Document parameters
    \title{Sample SIR modelling}
    
    
    

    % Pygments definitions
    
\makeatletter
\def\PY@reset{\let\PY@it=\relax \let\PY@bf=\relax%
    \let\PY@ul=\relax \let\PY@tc=\relax%
    \let\PY@bc=\relax \let\PY@ff=\relax}
\def\PY@tok#1{\csname PY@tok@#1\endcsname}
\def\PY@toks#1+{\ifx\relax#1\empty\else%
    \PY@tok{#1}\expandafter\PY@toks\fi}
\def\PY@do#1{\PY@bc{\PY@tc{\PY@ul{%
    \PY@it{\PY@bf{\PY@ff{#1}}}}}}}
\def\PY#1#2{\PY@reset\PY@toks#1+\relax+\PY@do{#2}}

\expandafter\def\csname PY@tok@w\endcsname{\def\PY@tc##1{\textcolor[rgb]{0.73,0.73,0.73}{##1}}}
\expandafter\def\csname PY@tok@c\endcsname{\let\PY@it=\textit\def\PY@tc##1{\textcolor[rgb]{0.25,0.50,0.50}{##1}}}
\expandafter\def\csname PY@tok@cp\endcsname{\def\PY@tc##1{\textcolor[rgb]{0.74,0.48,0.00}{##1}}}
\expandafter\def\csname PY@tok@k\endcsname{\let\PY@bf=\textbf\def\PY@tc##1{\textcolor[rgb]{0.00,0.50,0.00}{##1}}}
\expandafter\def\csname PY@tok@kp\endcsname{\def\PY@tc##1{\textcolor[rgb]{0.00,0.50,0.00}{##1}}}
\expandafter\def\csname PY@tok@kt\endcsname{\def\PY@tc##1{\textcolor[rgb]{0.69,0.00,0.25}{##1}}}
\expandafter\def\csname PY@tok@o\endcsname{\def\PY@tc##1{\textcolor[rgb]{0.40,0.40,0.40}{##1}}}
\expandafter\def\csname PY@tok@ow\endcsname{\let\PY@bf=\textbf\def\PY@tc##1{\textcolor[rgb]{0.67,0.13,1.00}{##1}}}
\expandafter\def\csname PY@tok@nb\endcsname{\def\PY@tc##1{\textcolor[rgb]{0.00,0.50,0.00}{##1}}}
\expandafter\def\csname PY@tok@nf\endcsname{\def\PY@tc##1{\textcolor[rgb]{0.00,0.00,1.00}{##1}}}
\expandafter\def\csname PY@tok@nc\endcsname{\let\PY@bf=\textbf\def\PY@tc##1{\textcolor[rgb]{0.00,0.00,1.00}{##1}}}
\expandafter\def\csname PY@tok@nn\endcsname{\let\PY@bf=\textbf\def\PY@tc##1{\textcolor[rgb]{0.00,0.00,1.00}{##1}}}
\expandafter\def\csname PY@tok@ne\endcsname{\let\PY@bf=\textbf\def\PY@tc##1{\textcolor[rgb]{0.82,0.25,0.23}{##1}}}
\expandafter\def\csname PY@tok@nv\endcsname{\def\PY@tc##1{\textcolor[rgb]{0.10,0.09,0.49}{##1}}}
\expandafter\def\csname PY@tok@no\endcsname{\def\PY@tc##1{\textcolor[rgb]{0.53,0.00,0.00}{##1}}}
\expandafter\def\csname PY@tok@nl\endcsname{\def\PY@tc##1{\textcolor[rgb]{0.63,0.63,0.00}{##1}}}
\expandafter\def\csname PY@tok@ni\endcsname{\let\PY@bf=\textbf\def\PY@tc##1{\textcolor[rgb]{0.60,0.60,0.60}{##1}}}
\expandafter\def\csname PY@tok@na\endcsname{\def\PY@tc##1{\textcolor[rgb]{0.49,0.56,0.16}{##1}}}
\expandafter\def\csname PY@tok@nt\endcsname{\let\PY@bf=\textbf\def\PY@tc##1{\textcolor[rgb]{0.00,0.50,0.00}{##1}}}
\expandafter\def\csname PY@tok@nd\endcsname{\def\PY@tc##1{\textcolor[rgb]{0.67,0.13,1.00}{##1}}}
\expandafter\def\csname PY@tok@s\endcsname{\def\PY@tc##1{\textcolor[rgb]{0.73,0.13,0.13}{##1}}}
\expandafter\def\csname PY@tok@sd\endcsname{\let\PY@it=\textit\def\PY@tc##1{\textcolor[rgb]{0.73,0.13,0.13}{##1}}}
\expandafter\def\csname PY@tok@si\endcsname{\let\PY@bf=\textbf\def\PY@tc##1{\textcolor[rgb]{0.73,0.40,0.53}{##1}}}
\expandafter\def\csname PY@tok@se\endcsname{\let\PY@bf=\textbf\def\PY@tc##1{\textcolor[rgb]{0.73,0.40,0.13}{##1}}}
\expandafter\def\csname PY@tok@sr\endcsname{\def\PY@tc##1{\textcolor[rgb]{0.73,0.40,0.53}{##1}}}
\expandafter\def\csname PY@tok@ss\endcsname{\def\PY@tc##1{\textcolor[rgb]{0.10,0.09,0.49}{##1}}}
\expandafter\def\csname PY@tok@sx\endcsname{\def\PY@tc##1{\textcolor[rgb]{0.00,0.50,0.00}{##1}}}
\expandafter\def\csname PY@tok@m\endcsname{\def\PY@tc##1{\textcolor[rgb]{0.40,0.40,0.40}{##1}}}
\expandafter\def\csname PY@tok@gh\endcsname{\let\PY@bf=\textbf\def\PY@tc##1{\textcolor[rgb]{0.00,0.00,0.50}{##1}}}
\expandafter\def\csname PY@tok@gu\endcsname{\let\PY@bf=\textbf\def\PY@tc##1{\textcolor[rgb]{0.50,0.00,0.50}{##1}}}
\expandafter\def\csname PY@tok@gd\endcsname{\def\PY@tc##1{\textcolor[rgb]{0.63,0.00,0.00}{##1}}}
\expandafter\def\csname PY@tok@gi\endcsname{\def\PY@tc##1{\textcolor[rgb]{0.00,0.63,0.00}{##1}}}
\expandafter\def\csname PY@tok@gr\endcsname{\def\PY@tc##1{\textcolor[rgb]{1.00,0.00,0.00}{##1}}}
\expandafter\def\csname PY@tok@ge\endcsname{\let\PY@it=\textit}
\expandafter\def\csname PY@tok@gs\endcsname{\let\PY@bf=\textbf}
\expandafter\def\csname PY@tok@gp\endcsname{\let\PY@bf=\textbf\def\PY@tc##1{\textcolor[rgb]{0.00,0.00,0.50}{##1}}}
\expandafter\def\csname PY@tok@go\endcsname{\def\PY@tc##1{\textcolor[rgb]{0.53,0.53,0.53}{##1}}}
\expandafter\def\csname PY@tok@gt\endcsname{\def\PY@tc##1{\textcolor[rgb]{0.00,0.27,0.87}{##1}}}
\expandafter\def\csname PY@tok@err\endcsname{\def\PY@bc##1{\setlength{\fboxsep}{0pt}\fcolorbox[rgb]{1.00,0.00,0.00}{1,1,1}{\strut ##1}}}
\expandafter\def\csname PY@tok@kc\endcsname{\let\PY@bf=\textbf\def\PY@tc##1{\textcolor[rgb]{0.00,0.50,0.00}{##1}}}
\expandafter\def\csname PY@tok@kd\endcsname{\let\PY@bf=\textbf\def\PY@tc##1{\textcolor[rgb]{0.00,0.50,0.00}{##1}}}
\expandafter\def\csname PY@tok@kn\endcsname{\let\PY@bf=\textbf\def\PY@tc##1{\textcolor[rgb]{0.00,0.50,0.00}{##1}}}
\expandafter\def\csname PY@tok@kr\endcsname{\let\PY@bf=\textbf\def\PY@tc##1{\textcolor[rgb]{0.00,0.50,0.00}{##1}}}
\expandafter\def\csname PY@tok@bp\endcsname{\def\PY@tc##1{\textcolor[rgb]{0.00,0.50,0.00}{##1}}}
\expandafter\def\csname PY@tok@fm\endcsname{\def\PY@tc##1{\textcolor[rgb]{0.00,0.00,1.00}{##1}}}
\expandafter\def\csname PY@tok@vc\endcsname{\def\PY@tc##1{\textcolor[rgb]{0.10,0.09,0.49}{##1}}}
\expandafter\def\csname PY@tok@vg\endcsname{\def\PY@tc##1{\textcolor[rgb]{0.10,0.09,0.49}{##1}}}
\expandafter\def\csname PY@tok@vi\endcsname{\def\PY@tc##1{\textcolor[rgb]{0.10,0.09,0.49}{##1}}}
\expandafter\def\csname PY@tok@vm\endcsname{\def\PY@tc##1{\textcolor[rgb]{0.10,0.09,0.49}{##1}}}
\expandafter\def\csname PY@tok@sa\endcsname{\def\PY@tc##1{\textcolor[rgb]{0.73,0.13,0.13}{##1}}}
\expandafter\def\csname PY@tok@sb\endcsname{\def\PY@tc##1{\textcolor[rgb]{0.73,0.13,0.13}{##1}}}
\expandafter\def\csname PY@tok@sc\endcsname{\def\PY@tc##1{\textcolor[rgb]{0.73,0.13,0.13}{##1}}}
\expandafter\def\csname PY@tok@dl\endcsname{\def\PY@tc##1{\textcolor[rgb]{0.73,0.13,0.13}{##1}}}
\expandafter\def\csname PY@tok@s2\endcsname{\def\PY@tc##1{\textcolor[rgb]{0.73,0.13,0.13}{##1}}}
\expandafter\def\csname PY@tok@sh\endcsname{\def\PY@tc##1{\textcolor[rgb]{0.73,0.13,0.13}{##1}}}
\expandafter\def\csname PY@tok@s1\endcsname{\def\PY@tc##1{\textcolor[rgb]{0.73,0.13,0.13}{##1}}}
\expandafter\def\csname PY@tok@mb\endcsname{\def\PY@tc##1{\textcolor[rgb]{0.40,0.40,0.40}{##1}}}
\expandafter\def\csname PY@tok@mf\endcsname{\def\PY@tc##1{\textcolor[rgb]{0.40,0.40,0.40}{##1}}}
\expandafter\def\csname PY@tok@mh\endcsname{\def\PY@tc##1{\textcolor[rgb]{0.40,0.40,0.40}{##1}}}
\expandafter\def\csname PY@tok@mi\endcsname{\def\PY@tc##1{\textcolor[rgb]{0.40,0.40,0.40}{##1}}}
\expandafter\def\csname PY@tok@il\endcsname{\def\PY@tc##1{\textcolor[rgb]{0.40,0.40,0.40}{##1}}}
\expandafter\def\csname PY@tok@mo\endcsname{\def\PY@tc##1{\textcolor[rgb]{0.40,0.40,0.40}{##1}}}
\expandafter\def\csname PY@tok@ch\endcsname{\let\PY@it=\textit\def\PY@tc##1{\textcolor[rgb]{0.25,0.50,0.50}{##1}}}
\expandafter\def\csname PY@tok@cm\endcsname{\let\PY@it=\textit\def\PY@tc##1{\textcolor[rgb]{0.25,0.50,0.50}{##1}}}
\expandafter\def\csname PY@tok@cpf\endcsname{\let\PY@it=\textit\def\PY@tc##1{\textcolor[rgb]{0.25,0.50,0.50}{##1}}}
\expandafter\def\csname PY@tok@c1\endcsname{\let\PY@it=\textit\def\PY@tc##1{\textcolor[rgb]{0.25,0.50,0.50}{##1}}}
\expandafter\def\csname PY@tok@cs\endcsname{\let\PY@it=\textit\def\PY@tc##1{\textcolor[rgb]{0.25,0.50,0.50}{##1}}}

\def\PYZbs{\char`\\}
\def\PYZus{\char`\_}
\def\PYZob{\char`\{}
\def\PYZcb{\char`\}}
\def\PYZca{\char`\^}
\def\PYZam{\char`\&}
\def\PYZlt{\char`\<}
\def\PYZgt{\char`\>}
\def\PYZsh{\char`\#}
\def\PYZpc{\char`\%}
\def\PYZdl{\char`\$}
\def\PYZhy{\char`\-}
\def\PYZsq{\char`\'}
\def\PYZdq{\char`\"}
\def\PYZti{\char`\~}
% for compatibility with earlier versions
\def\PYZat{@}
\def\PYZlb{[}
\def\PYZrb{]}
\makeatother


    % Exact colors from NB
    \definecolor{incolor}{rgb}{0.0, 0.0, 0.5}
    \definecolor{outcolor}{rgb}{0.545, 0.0, 0.0}



    
    % Prevent overflowing lines due to hard-to-break entities
    \sloppy 
    % Setup hyperref package
    \hypersetup{
      breaklinks=true,  % so long urls are correctly broken across lines
      colorlinks=true,
      urlcolor=urlcolor,
      linkcolor=linkcolor,
      citecolor=citecolor,
      }
    % Slightly bigger margins than the latex defaults
    
    \geometry{verbose,tmargin=1in,bmargin=1in,lmargin=1in,rmargin=1in}
    
    

    \begin{document}
    
    
    \maketitle
    
    

    
    \section{SIR Modelling}\label{sir-modelling}

\subsection{Nhật ký}\label{nhuxe2t-ky}

\begin{itemize}
\tightlist
\item
  Lên ý tưởng :

  \begin{itemize}
  \tightlist
  \item
    làm về covid - có những hướng gì trong hiện tại : so sánh với các
    bệnh dịch cũ (cúm heo, cúm gà, cúm Spanish 1918) (literature
    review), dự đoán tình hình covid sẽ như thế nào cho thời gian tới,
    giữa việc có social-distancing và không có sẽ ảnh hưởng như thế nào
    tới tình phát triển bệnh dịch.
  \item
    \textbf{Chọn đề tài so sánh giữa nghiên cứu sự ảnh hưởng có và không
    có social-distancing}
  \end{itemize}
\item
  Phương pháp thực hiện nghiên cứu :

  \begin{itemize}
  \tightlist
  \item
    Phân tích ý tưởng :

    \begin{itemize}
    \tightlist
    \item
      Định nghĩa social-distancing, những điểm khác nhau nào khi có và
      không có social-distancing.
    \item
      Sự ảnh hưởng lên những đối tượng nào? death population, infected
      population growing trend
    \item
      Muốn chứng minh việc có social-distancing sẽ tốt cho xã hội trong
      việc giảm số lượng người chết và tạo 1 koảng thời gian giảm sốc
      cho cơ sở hạ tầng bắt nhịp cùng với dịch bệnh. Vậy thì chúng ta
      phải lên những bước nào để chứng minh cho hypothesis này?
    \end{itemize}
  \item
    Nghiên cứu ảnh hưởng của việc social-distancing một cách
    mathematical:

    \begin{itemize}
    \tightlist
    \item
      Xây dựng những hypothesis below dựa trên \textbf{intuition} của
      bản thân
    \item
      \(\uparrow\) social-distancing \(\rightarrow\) \(\downarrow\)
      tranmission rate. (statistic?)
    \item
      \(\downarrow\) tranmission rate \(\rightarrow\) \(\downarrow\) in
      newly infected population.
    \item
      \(\downarrow\) in newly infected population \(\rightarrow\) leaves
      time for medical infrastructure to keep up with the pandemic
      \(\rightarrow\) \(\downarrow\) death population in the long run.
    \end{itemize}
  \item
    Liên tưởng tới việc so sánh:

    \begin{itemize}
    \tightlist
    \item
      Dữ liệu của các nước có vs các nước ko có social-distancing (Ý, Mỹ
      so với Nhật Bản, Hàn Quốc)
    \item
      Dựa trên dữ liệu hiện tại của 1 nước, áp dụng một mô hình dịch
      bệnh và so sánh sự khác biệt nếu có của social-distancing (giảm
      tranmission rate - tỉ lệ lây lan)
    \end{itemize}
  \item
    Chọn lựa hướng đi dựa trên background của bản thân:

    \begin{itemize}
    \tightlist
    \item
      Nếu chúng ta có 1 decent mathematical background + coding +
      English + tí biology
    \item
      \(\rightarrow\) \textbf{Chọn modelling}
    \end{itemize}
  \item
    Có các loại models nào:

    \begin{itemize}
    \tightlist
    \item
      Dựa trên
      https://en.wikipedia.org/wiki/Mathematical\_modelling\_of\_infectious\_disease
      chúng ta có Stochastic và Deterministic
    \item
      Stochastic sẽ dùng các random variable, lượng kiến thức sẽ liên
      quan tới xác suất thống kê và probability theory.
    \item
      Deterministic là đa số compartmentals models tức là những mô hình
      chia dân số ra theo nhóm và phân tích quá trình trao đổi các cá
      thể giữa các nhóm.
    \item
      \(\rightarrow\) \textbf{Chọn compartmental models in epidemiology}
      (dễ hiểu hơn + yêu cầu toán không cao)
    \item
      Các loại compartmental models : SIR (1927) and its modified
      versions.
    \item
      \(\rightarrow\) \textbf{Chọn SIR}
    \end{itemize}
  \end{itemize}
\item
  Resources that could be useful:

  \begin{itemize}
  \tightlist
  \item
    https://github.com/midas-network/COVID-19/tree/master/parameter\_estimates/2019\_novel\_coronavirus
  \item
    https://en.wikipedia.org/wiki/Compartmental\_models\_in\_epidemiology
  \item
    https://en.wikipedia.org/wiki/Mathematical\_modelling\_of\_infectious\_disease
  \end{itemize}
\end{itemize}

\subsection{\#\# Ý nghĩa đạo hàm}\label{y-nghia-ux111ao-ham}

\subsection{Mô hình và tham số}\label{muxf4-hinh-va-tham-suxf4}

\begin{itemize}
\tightlist
\item
  Mô hình SIR là một trong những mô hình toán học đơn giản nhất để phân
  tích, dự đoán cho những bệnh truyền nhiễm. Ý tưởng của mô hình là chia
  quần thể (population) thành các nhóm (compartments) riêng biệt tương
  ứng với từng giai đoạn của bệnh. Sau đó chúng ta phân tích những sự
  trao đổi các cá thể của những compartments riêng biệt này
\item
  Tên của mô hình SIR có ý nghĩa của 3 compartments

  \begin{itemize}
  \tightlist
  \item
    S : Susceptible - phần dân số có khả năng nhiễm bệnh.
  \item
    I : Infected - phần dân số đã nhiễm bệnh.
  \item
    R : Recovered and have immunity + fatal - phần dân số đã hồi phục và
    miễn dịch cộng với phần dân số tử vong
  \item
    Chú ý mỗi compartment là một biến theo thời gian \(S(t),I(t),R(t)\)
  \end{itemize}
\item
  Mô hình SIR có những tham số liên quan sau:

  \begin{itemize}
  \tightlist
  \item
    \(\beta\) : tham số thể hiện các tiếp xúc lây nhiễm thành công của 1
    cá thể nhiễm bệnh (Infected) và các thể chưa nhiễm bệnh
    (Susceptible) trên một đơn vị thời gian.
  \item
    \(\gamma\) : tham số thể hiện tỉ lệ hồi phục trên trung bình. Chú ý
    \(\frac{1}{\gamma}\) là khoảng thời gian một cá thể nhiễm bệnh hồi
    phục.
  \end{itemize}
\end{itemize}

\subsection{Sự phân hóa thay đổi trong mỗi
nhóm}\label{sux1b0-phuxe2n-hoa-thay-ux111uxf4i-trong-muxf4i-nhom}

\begin{itemize}
\tightlist
\item
  Phương trình vi phân mô hình hóa sự thay đổi số lượng cá thể trong mỗi
  compartment:
\end{itemize}

\begin{align}
\frac{dS}{dt} &= -\frac{\beta SI}{N} \\
\frac{dI}{dt} &= \frac{\beta SI}{N} - \gamma I\\
\frac{dR}{dt} &= \gamma I
\end{align}

\begin{itemize}
\tightlist
\item
  Vì tổng dân số là \(N = S(t) + I(t) + R(t)\) nên chúng ta thấy được
  tổng các thay đổi trong các compartment được bảo toàn

  \begin{equation}
  \frac{dS}{dt}+\frac{dI}{dt}+\frac{dR}{dt} = 0
  \end{equation}
\end{itemize}

\subsection{Hệ số lây nhiễm cơ
bản}\label{huxea-suxf4-luxe2y-nhiuxeam-cux1a1-ban}

\begin{itemize}
\tightlist
\item
  Đại lượng đặc biệt \(R_o\) được gọi là hệ số lây nhiễm cơ bản được
  định nghĩa bởi

  \begin{equation}
  R_o = \frac{\beta}{\gamma}
  \end{equation}
\item
  Chúng ta thay hệ số này vào phương trình của \(\frac{dI}{dt}\), ta
  được

  \begin{equation}
  \frac{dI}{dt} = \frac{\beta SI \gamma}{N \gamma} - \gamma I = (R_o \frac{S}{N}-1)\gamma I
  \end{equation}
\item
  Chú ý rằng khi thời giam \(t=0\) tức chưa xảy ra dịch bệnh thì
  \(S(0) = N\), chúng ta phân tích tại thời điểm \(t=0\), nếu

  \begin{equation}
  R_o > \frac{N}{S(0)} = 1 \longrightarrow \frac{dI}{dt} > 0
  \end{equation}
\item
  Ý nghĩa của tham số \(\frac{dI}{dt}\) biểu hiện cho sự thay đổi (tăng
  giảm) của nhóm bị nhiễm bệnh qua thời gian, vậy nếu ngay từ thời gian
  bắt đầu \(\frac{dI}{dt}(0)>0\) thì bệnh dịch tất nhiên sẽ xảy ra.
  Ngược lại, nếu tham số đó tại \(t = 0\) là âm, thì bệnh dịch sẽ không
  lây lan.
\end{itemize}

    \begin{Verbatim}[commandchars=\\\{\}]
{\color{incolor}In [{\color{incolor}6}]:} \PY{k+kn}{import} \PY{n+nn}{numpy} \PY{k}{as} \PY{n+nn}{np}
        \PY{k+kn}{from} \PY{n+nn}{scipy}\PY{n+nn}{.}\PY{n+nn}{integrate} \PY{k}{import} \PY{n}{odeint}
        \PY{k+kn}{import} \PY{n+nn}{matplotlib}\PY{n+nn}{.}\PY{n+nn}{pyplot} \PY{k}{as} \PY{n+nn}{plt}
        
        \PY{c+c1}{\PYZsh{} Total population, N.}
        \PY{n}{N} \PY{o}{=} \PY{l+m+mi}{7800}
        \PY{c+c1}{\PYZsh{} Initial number of infected and recovered individuals, I0 and R0.}
        \PY{n}{I0}\PY{p}{,} \PY{n}{R0} \PY{o}{=} \PY{l+m+mi}{1}\PY{p}{,} \PY{l+m+mi}{0}
        \PY{c+c1}{\PYZsh{} Everyone else, S0, is susceptible to infection initially.}
        \PY{n}{S0} \PY{o}{=} \PY{n}{N} \PY{o}{\PYZhy{}} \PY{n}{I0} \PY{o}{\PYZhy{}} \PY{n}{R0}
        \PY{c+c1}{\PYZsh{} Contact rate, beta, and mean recovery rate, gamma, (in 1/days).}
        \PY{n}{beta}\PY{p}{,} \PY{n}{gamma} \PY{o}{=} \PY{l+m+mf}{0.7}\PY{p}{,} \PY{l+m+mf}{0.2} 
        \PY{c+c1}{\PYZsh{} A grid of time points (in days)}
        \PY{n}{t} \PY{o}{=} \PY{n}{np}\PY{o}{.}\PY{n}{linspace}\PY{p}{(}\PY{l+m+mi}{0}\PY{p}{,} \PY{l+m+mi}{160}\PY{p}{,} \PY{l+m+mi}{160}\PY{p}{)}
        
        \PY{c+c1}{\PYZsh{} The SIR model differential equations.}
        \PY{k}{def} \PY{n+nf}{deriv}\PY{p}{(}\PY{n}{y}\PY{p}{,} \PY{n}{t}\PY{p}{,} \PY{n}{N}\PY{p}{,} \PY{n}{beta}\PY{p}{,} \PY{n}{gamma}\PY{p}{)}\PY{p}{:}
            \PY{n}{S}\PY{p}{,} \PY{n}{I}\PY{p}{,} \PY{n}{R} \PY{o}{=} \PY{n}{y}
            \PY{n}{dSdt} \PY{o}{=} \PY{o}{\PYZhy{}}\PY{n}{beta} \PY{o}{*} \PY{n}{S} \PY{o}{*} \PY{n}{I} \PY{o}{/} \PY{n}{N}
            \PY{n}{dIdt} \PY{o}{=} \PY{n}{beta} \PY{o}{*} \PY{n}{S} \PY{o}{*} \PY{n}{I} \PY{o}{/} \PY{n}{N} \PY{o}{\PYZhy{}} \PY{n}{gamma} \PY{o}{*} \PY{n}{I}
            \PY{n}{dRdt} \PY{o}{=} \PY{n}{gamma} \PY{o}{*} \PY{n}{I}
            \PY{k}{return} \PY{n}{dSdt}\PY{p}{,} \PY{n}{dIdt}\PY{p}{,} \PY{n}{dRdt}
        
        \PY{c+c1}{\PYZsh{} Initial conditions vector}
        \PY{n}{y0} \PY{o}{=} \PY{n}{S0}\PY{p}{,} \PY{n}{I0}\PY{p}{,} \PY{n}{R0}
        \PY{c+c1}{\PYZsh{} Integrate the SIR equations over the time grid, t.}
        \PY{n}{ret} \PY{o}{=} \PY{n}{odeint}\PY{p}{(}\PY{n}{deriv}\PY{p}{,} \PY{n}{y0}\PY{p}{,} \PY{n}{t}\PY{p}{,} \PY{n}{args}\PY{o}{=}\PY{p}{(}\PY{n}{N}\PY{p}{,} \PY{n}{beta}\PY{p}{,} \PY{n}{gamma}\PY{p}{)}\PY{p}{)}
        \PY{n}{S}\PY{p}{,} \PY{n}{I}\PY{p}{,} \PY{n}{R} \PY{o}{=} \PY{n}{ret}\PY{o}{.}\PY{n}{T}
        
        \PY{c+c1}{\PYZsh{} Plot the data on three separate curves for S(t), I(t) and R(t)}
        \PY{n}{fig} \PY{o}{=} \PY{n}{plt}\PY{o}{.}\PY{n}{figure}\PY{p}{(}\PY{n}{facecolor}\PY{o}{=}\PY{l+s+s1}{\PYZsq{}}\PY{l+s+s1}{w}\PY{l+s+s1}{\PYZsq{}}\PY{p}{,}\PY{n}{figsize}\PY{o}{=}\PY{p}{(}\PY{l+m+mi}{20}\PY{p}{,}\PY{l+m+mi}{10}\PY{p}{)}\PY{p}{)}
        \PY{n}{ax} \PY{o}{=} \PY{n}{fig}\PY{o}{.}\PY{n}{add\PYZus{}subplot}\PY{p}{(}\PY{l+m+mi}{111}\PY{p}{,} \PY{n}{axisbelow}\PY{o}{=}\PY{k+kc}{True}\PY{p}{)}
        \PY{n}{ax}\PY{o}{.}\PY{n}{plot}\PY{p}{(}\PY{n}{t}\PY{p}{,} \PY{n}{S}\PY{o}{/}\PY{n}{N}\PY{p}{,} \PY{l+s+s1}{\PYZsq{}}\PY{l+s+s1}{b}\PY{l+s+s1}{\PYZsq{}}\PY{p}{,} \PY{n}{alpha}\PY{o}{=}\PY{l+m+mf}{0.5}\PY{p}{,} \PY{n}{lw}\PY{o}{=}\PY{l+m+mi}{2}\PY{p}{,} \PY{n}{label}\PY{o}{=}\PY{l+s+s1}{\PYZsq{}}\PY{l+s+s1}{Susceptible}\PY{l+s+s1}{\PYZsq{}}\PY{p}{)}
        \PY{n}{ax}\PY{o}{.}\PY{n}{plot}\PY{p}{(}\PY{n}{t}\PY{p}{,} \PY{n}{I}\PY{o}{/}\PY{n}{N}\PY{p}{,} \PY{l+s+s1}{\PYZsq{}}\PY{l+s+s1}{r}\PY{l+s+s1}{\PYZsq{}}\PY{p}{,} \PY{n}{alpha}\PY{o}{=}\PY{l+m+mf}{0.5}\PY{p}{,} \PY{n}{lw}\PY{o}{=}\PY{l+m+mi}{2}\PY{p}{,} \PY{n}{label}\PY{o}{=}\PY{l+s+s1}{\PYZsq{}}\PY{l+s+s1}{Infected}\PY{l+s+s1}{\PYZsq{}}\PY{p}{)}
        \PY{n}{ax}\PY{o}{.}\PY{n}{plot}\PY{p}{(}\PY{n}{t}\PY{p}{,} \PY{n}{R}\PY{o}{/}\PY{n}{N}\PY{p}{,} \PY{l+s+s1}{\PYZsq{}}\PY{l+s+s1}{g}\PY{l+s+s1}{\PYZsq{}}\PY{p}{,} \PY{n}{alpha}\PY{o}{=}\PY{l+m+mf}{0.5}\PY{p}{,} \PY{n}{lw}\PY{o}{=}\PY{l+m+mi}{2}\PY{p}{,} \PY{n}{label}\PY{o}{=}\PY{l+s+s1}{\PYZsq{}}\PY{l+s+s1}{Recovered with immunity}\PY{l+s+s1}{\PYZsq{}}\PY{p}{)}
        \PY{n}{ax}\PY{o}{.}\PY{n}{set\PYZus{}xlabel}\PY{p}{(}\PY{l+s+s1}{\PYZsq{}}\PY{l+s+s1}{Time /days}\PY{l+s+s1}{\PYZsq{}}\PY{p}{)}
        \PY{n}{ax}\PY{o}{.}\PY{n}{set\PYZus{}ylabel}\PY{p}{(}\PY{l+s+s1}{\PYZsq{}}\PY{l+s+s1}{Number (1000s)}\PY{l+s+s1}{\PYZsq{}}\PY{p}{)}
        \PY{n}{ax}\PY{o}{.}\PY{n}{set\PYZus{}ylim}\PY{p}{(}\PY{l+m+mi}{0}\PY{p}{,}\PY{l+m+mf}{1.2}\PY{p}{)}
        \PY{n}{ax}\PY{o}{.}\PY{n}{yaxis}\PY{o}{.}\PY{n}{set\PYZus{}tick\PYZus{}params}\PY{p}{(}\PY{n}{length}\PY{o}{=}\PY{l+m+mi}{0}\PY{p}{)}
        \PY{n}{ax}\PY{o}{.}\PY{n}{xaxis}\PY{o}{.}\PY{n}{set\PYZus{}tick\PYZus{}params}\PY{p}{(}\PY{n}{length}\PY{o}{=}\PY{l+m+mi}{0}\PY{p}{)}
        \PY{n}{ax}\PY{o}{.}\PY{n}{grid}\PY{p}{(}\PY{n}{b}\PY{o}{=}\PY{k+kc}{True}\PY{p}{,} \PY{n}{which}\PY{o}{=}\PY{l+s+s1}{\PYZsq{}}\PY{l+s+s1}{major}\PY{l+s+s1}{\PYZsq{}}\PY{p}{,} \PY{n}{c}\PY{o}{=}\PY{l+s+s1}{\PYZsq{}}\PY{l+s+s1}{w}\PY{l+s+s1}{\PYZsq{}}\PY{p}{,} \PY{n}{lw}\PY{o}{=}\PY{l+m+mi}{2}\PY{p}{,} \PY{n}{ls}\PY{o}{=}\PY{l+s+s1}{\PYZsq{}}\PY{l+s+s1}{\PYZhy{}}\PY{l+s+s1}{\PYZsq{}}\PY{p}{)}
        \PY{n}{legend} \PY{o}{=} \PY{n}{ax}\PY{o}{.}\PY{n}{legend}\PY{p}{(}\PY{p}{)}
        \PY{n}{legend}\PY{o}{.}\PY{n}{get\PYZus{}frame}\PY{p}{(}\PY{p}{)}\PY{o}{.}\PY{n}{set\PYZus{}alpha}\PY{p}{(}\PY{l+m+mf}{0.5}\PY{p}{)}
        \PY{k}{for} \PY{n}{spine} \PY{o+ow}{in} \PY{p}{(}\PY{l+s+s1}{\PYZsq{}}\PY{l+s+s1}{top}\PY{l+s+s1}{\PYZsq{}}\PY{p}{,} \PY{l+s+s1}{\PYZsq{}}\PY{l+s+s1}{right}\PY{l+s+s1}{\PYZsq{}}\PY{p}{,} \PY{l+s+s1}{\PYZsq{}}\PY{l+s+s1}{bottom}\PY{l+s+s1}{\PYZsq{}}\PY{p}{,} \PY{l+s+s1}{\PYZsq{}}\PY{l+s+s1}{left}\PY{l+s+s1}{\PYZsq{}}\PY{p}{)}\PY{p}{:}
            \PY{n}{ax}\PY{o}{.}\PY{n}{spines}\PY{p}{[}\PY{n}{spine}\PY{p}{]}\PY{o}{.}\PY{n}{set\PYZus{}visible}\PY{p}{(}\PY{k+kc}{False}\PY{p}{)}
        \PY{n}{plt}\PY{o}{.}\PY{n}{show}\PY{p}{(}\PY{p}{)}
\end{Verbatim}


    \begin{center}
    \adjustimage{max size={0.9\linewidth}{0.9\paperheight}}{output_1_0.png}
    \end{center}
    { \hspace*{\fill} \\}
    
    \begin{Verbatim}[commandchars=\\\{\}]
{\color{incolor}In [{\color{incolor}9}]:} \PY{k+kn}{import} \PY{n+nn}{numpy} \PY{k}{as} \PY{n+nn}{np}
        \PY{k+kn}{from} \PY{n+nn}{scipy}\PY{n+nn}{.}\PY{n+nn}{integrate} \PY{k}{import} \PY{n}{odeint}
        \PY{k+kn}{import} \PY{n+nn}{matplotlib}\PY{n+nn}{.}\PY{n+nn}{pyplot} \PY{k}{as} \PY{n+nn}{plt}
        
        \PY{c+c1}{\PYZsh{} Total population, N.}
        \PY{n}{N} \PY{o}{=} \PY{l+m+mi}{7800}
        \PY{c+c1}{\PYZsh{} Initial number of infected and recovered individuals, I0 and R0.}
        \PY{n}{I0}\PY{p}{,} \PY{n}{R0} \PY{o}{=} \PY{l+m+mi}{1}\PY{p}{,} \PY{l+m+mi}{0}
        \PY{c+c1}{\PYZsh{} Everyone else, S0, is susceptible to infection initially.}
        \PY{n}{S0} \PY{o}{=} \PY{n}{N} \PY{o}{\PYZhy{}} \PY{n}{I0} \PY{o}{\PYZhy{}} \PY{n}{R0}
        \PY{c+c1}{\PYZsh{} Contact rate, beta, and mean recovery rate, gamma, (in 1/days).}
        \PY{n}{beta}\PY{p}{,} \PY{n}{gamma} \PY{o}{=} \PY{l+m+mf}{0.4}\PY{p}{,} \PY{l+m+mf}{0.2}
        \PY{c+c1}{\PYZsh{} A grid of time points (in days)}
        \PY{n}{t} \PY{o}{=} \PY{n}{np}\PY{o}{.}\PY{n}{linspace}\PY{p}{(}\PY{l+m+mi}{0}\PY{p}{,} \PY{l+m+mi}{160}\PY{p}{,} \PY{l+m+mi}{160}\PY{p}{)}
        
        \PY{c+c1}{\PYZsh{} The SIR model differential equations.}
        \PY{k}{def} \PY{n+nf}{deriv}\PY{p}{(}\PY{n}{y}\PY{p}{,} \PY{n}{t}\PY{p}{,} \PY{n}{N}\PY{p}{,} \PY{n}{beta}\PY{p}{,} \PY{n}{gamma}\PY{p}{)}\PY{p}{:}
            \PY{n}{S}\PY{p}{,} \PY{n}{I}\PY{p}{,} \PY{n}{R} \PY{o}{=} \PY{n}{y}
            \PY{n}{dSdt} \PY{o}{=} \PY{o}{\PYZhy{}}\PY{n}{beta} \PY{o}{*} \PY{n}{S} \PY{o}{*} \PY{n}{I} \PY{o}{/} \PY{n}{N}
            \PY{n}{dIdt} \PY{o}{=} \PY{n}{beta} \PY{o}{*} \PY{n}{S} \PY{o}{*} \PY{n}{I} \PY{o}{/} \PY{n}{N} \PY{o}{\PYZhy{}} \PY{n}{gamma} \PY{o}{*} \PY{n}{I}
            \PY{n}{dRdt} \PY{o}{=} \PY{n}{gamma} \PY{o}{*} \PY{n}{I}
            \PY{k}{return} \PY{n}{dSdt}\PY{p}{,} \PY{n}{dIdt}\PY{p}{,} \PY{n}{dRdt}
        
        \PY{c+c1}{\PYZsh{} Initial conditions vector}
        \PY{n}{y0} \PY{o}{=} \PY{n}{S0}\PY{p}{,} \PY{n}{I0}\PY{p}{,} \PY{n}{R0}
        \PY{c+c1}{\PYZsh{} Integrate the SIR equations over the time grid, t.}
        \PY{n}{ret} \PY{o}{=} \PY{n}{odeint}\PY{p}{(}\PY{n}{deriv}\PY{p}{,} \PY{n}{y0}\PY{p}{,} \PY{n}{t}\PY{p}{,} \PY{n}{args}\PY{o}{=}\PY{p}{(}\PY{n}{N}\PY{p}{,} \PY{n}{beta}\PY{p}{,} \PY{n}{gamma}\PY{p}{)}\PY{p}{)}
        \PY{n}{S}\PY{p}{,} \PY{n}{I}\PY{p}{,} \PY{n}{R} \PY{o}{=} \PY{n}{ret}\PY{o}{.}\PY{n}{T}
        
        \PY{c+c1}{\PYZsh{} Plot the data on three separate curves for S(t), I(t) and R(t)}
        \PY{n}{fig} \PY{o}{=} \PY{n}{plt}\PY{o}{.}\PY{n}{figure}\PY{p}{(}\PY{n}{facecolor}\PY{o}{=}\PY{l+s+s1}{\PYZsq{}}\PY{l+s+s1}{w}\PY{l+s+s1}{\PYZsq{}}\PY{p}{,}\PY{n}{figsize}\PY{o}{=}\PY{p}{(}\PY{l+m+mi}{20}\PY{p}{,}\PY{l+m+mi}{10}\PY{p}{)}\PY{p}{)}
        \PY{n}{ax} \PY{o}{=} \PY{n}{fig}\PY{o}{.}\PY{n}{add\PYZus{}subplot}\PY{p}{(}\PY{l+m+mi}{111}\PY{p}{,} \PY{n}{axisbelow}\PY{o}{=}\PY{k+kc}{True}\PY{p}{)}
        \PY{n}{ax}\PY{o}{.}\PY{n}{plot}\PY{p}{(}\PY{n}{t}\PY{p}{,} \PY{n}{S}\PY{o}{/}\PY{n}{N}\PY{p}{,} \PY{l+s+s1}{\PYZsq{}}\PY{l+s+s1}{b}\PY{l+s+s1}{\PYZsq{}}\PY{p}{,} \PY{n}{alpha}\PY{o}{=}\PY{l+m+mf}{0.5}\PY{p}{,} \PY{n}{lw}\PY{o}{=}\PY{l+m+mi}{2}\PY{p}{,} \PY{n}{label}\PY{o}{=}\PY{l+s+s1}{\PYZsq{}}\PY{l+s+s1}{Susceptible}\PY{l+s+s1}{\PYZsq{}}\PY{p}{)}
        \PY{n}{ax}\PY{o}{.}\PY{n}{plot}\PY{p}{(}\PY{n}{t}\PY{p}{,} \PY{n}{I}\PY{o}{/}\PY{n}{N}\PY{p}{,} \PY{l+s+s1}{\PYZsq{}}\PY{l+s+s1}{r}\PY{l+s+s1}{\PYZsq{}}\PY{p}{,} \PY{n}{alpha}\PY{o}{=}\PY{l+m+mf}{0.5}\PY{p}{,} \PY{n}{lw}\PY{o}{=}\PY{l+m+mi}{2}\PY{p}{,} \PY{n}{label}\PY{o}{=}\PY{l+s+s1}{\PYZsq{}}\PY{l+s+s1}{Infected}\PY{l+s+s1}{\PYZsq{}}\PY{p}{)}
        \PY{n}{ax}\PY{o}{.}\PY{n}{plot}\PY{p}{(}\PY{n}{t}\PY{p}{,} \PY{n}{R}\PY{o}{/}\PY{n}{N}\PY{p}{,} \PY{l+s+s1}{\PYZsq{}}\PY{l+s+s1}{g}\PY{l+s+s1}{\PYZsq{}}\PY{p}{,} \PY{n}{alpha}\PY{o}{=}\PY{l+m+mf}{0.5}\PY{p}{,} \PY{n}{lw}\PY{o}{=}\PY{l+m+mi}{2}\PY{p}{,} \PY{n}{label}\PY{o}{=}\PY{l+s+s1}{\PYZsq{}}\PY{l+s+s1}{Recovered with immunity}\PY{l+s+s1}{\PYZsq{}}\PY{p}{)}
        \PY{n}{ax}\PY{o}{.}\PY{n}{set\PYZus{}xlabel}\PY{p}{(}\PY{l+s+s1}{\PYZsq{}}\PY{l+s+s1}{Time /days}\PY{l+s+s1}{\PYZsq{}}\PY{p}{)}
        \PY{n}{ax}\PY{o}{.}\PY{n}{set\PYZus{}ylabel}\PY{p}{(}\PY{l+s+s1}{\PYZsq{}}\PY{l+s+s1}{Number (1000s)}\PY{l+s+s1}{\PYZsq{}}\PY{p}{)}
        \PY{n}{ax}\PY{o}{.}\PY{n}{set\PYZus{}ylim}\PY{p}{(}\PY{l+m+mi}{0}\PY{p}{,}\PY{l+m+mf}{1.2}\PY{p}{)}
        \PY{n}{ax}\PY{o}{.}\PY{n}{yaxis}\PY{o}{.}\PY{n}{set\PYZus{}tick\PYZus{}params}\PY{p}{(}\PY{n}{length}\PY{o}{=}\PY{l+m+mi}{0}\PY{p}{)}
        \PY{n}{ax}\PY{o}{.}\PY{n}{xaxis}\PY{o}{.}\PY{n}{set\PYZus{}tick\PYZus{}params}\PY{p}{(}\PY{n}{length}\PY{o}{=}\PY{l+m+mi}{0}\PY{p}{)}
        \PY{n}{ax}\PY{o}{.}\PY{n}{grid}\PY{p}{(}\PY{n}{b}\PY{o}{=}\PY{k+kc}{True}\PY{p}{,} \PY{n}{which}\PY{o}{=}\PY{l+s+s1}{\PYZsq{}}\PY{l+s+s1}{major}\PY{l+s+s1}{\PYZsq{}}\PY{p}{,} \PY{n}{c}\PY{o}{=}\PY{l+s+s1}{\PYZsq{}}\PY{l+s+s1}{w}\PY{l+s+s1}{\PYZsq{}}\PY{p}{,} \PY{n}{lw}\PY{o}{=}\PY{l+m+mi}{2}\PY{p}{,} \PY{n}{ls}\PY{o}{=}\PY{l+s+s1}{\PYZsq{}}\PY{l+s+s1}{\PYZhy{}}\PY{l+s+s1}{\PYZsq{}}\PY{p}{)}
        \PY{n}{legend} \PY{o}{=} \PY{n}{ax}\PY{o}{.}\PY{n}{legend}\PY{p}{(}\PY{p}{)}
        \PY{n}{legend}\PY{o}{.}\PY{n}{get\PYZus{}frame}\PY{p}{(}\PY{p}{)}\PY{o}{.}\PY{n}{set\PYZus{}alpha}\PY{p}{(}\PY{l+m+mf}{0.5}\PY{p}{)}
        \PY{k}{for} \PY{n}{spine} \PY{o+ow}{in} \PY{p}{(}\PY{l+s+s1}{\PYZsq{}}\PY{l+s+s1}{top}\PY{l+s+s1}{\PYZsq{}}\PY{p}{,} \PY{l+s+s1}{\PYZsq{}}\PY{l+s+s1}{right}\PY{l+s+s1}{\PYZsq{}}\PY{p}{,} \PY{l+s+s1}{\PYZsq{}}\PY{l+s+s1}{bottom}\PY{l+s+s1}{\PYZsq{}}\PY{p}{,} \PY{l+s+s1}{\PYZsq{}}\PY{l+s+s1}{left}\PY{l+s+s1}{\PYZsq{}}\PY{p}{)}\PY{p}{:}
            \PY{n}{ax}\PY{o}{.}\PY{n}{spines}\PY{p}{[}\PY{n}{spine}\PY{p}{]}\PY{o}{.}\PY{n}{set\PYZus{}visible}\PY{p}{(}\PY{k+kc}{False}\PY{p}{)}
        \PY{n}{plt}\PY{o}{.}\PY{n}{show}\PY{p}{(}\PY{p}{)}
\end{Verbatim}


    \begin{center}
    \adjustimage{max size={0.9\linewidth}{0.9\paperheight}}{output_2_0.png}
    \end{center}
    { \hspace*{\fill} \\}
    

    % Add a bibliography block to the postdoc
    
    
    
    \end{document}
